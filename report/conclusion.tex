\section{Conclusion} \

    During this internship we successfully formalized in \Lean the notion of groups of Lie type using $BN$-pairs. This allowed formalize quantities of results about groups with $BN$-pair. To do so, we introduced the new concept $BN\pi$-triplets. We also provided a concrete construction of the $BN$-pair over $GL_n\left( F \right)$, which allowed us to endow the groups $PGL_n(F)$, $SL_n(F)$ and $PSL_n(F)$ with such structure. 

    We additionally created a framework which can be used to demonstrate the simplicity of simple groups with $BN$-pairs and started applying it to $PSL_n(F)$.
    However, the above mentioned framework is currently relying on unproven theorem and additional work in the field of Coxeter groups theory will be needed to add it to \lean{mathlib}. All the necessary results can be found in §1 of \cite{bourbaki_groupes_2007} and their list in available on \href{https://github.com/corent1234/BNpairs}{github}.

    Nonetheless, once these obstacles have been removed the code produced during this internship will be able to be used a foundation to the theory of groups of Lie type. Therefore, future works on the formalization of the simplicity of such groups could happen in short amounts of time bringing the objective of formalizing the CFSG closer than ever.

    This internship has been a very enriching experience which allowed me broaden my mathematical culture and confirmed my interest in fundamental mathematics. Discovering and working with \Lean proved to be both mathematically enriching and extremely entertaining. I will probably continue to formalize mathematics on my free time. Additionally, meeting and talking with people with very different mathematical backgrounds which was really stimulating. \vspace{-.5em} 
